% Conserta o indent para alinhar conforme padrão de normalização
\addtocontents{toc}{\cftsetindents{chapter}{1.5cm}{3em}}
\addtocontents{toc}{\cftsetindents{section}{0cm}{1.5cm}}
\addtocontents{toc}{\cftsetindents{subsection}{0cm}{1.5cm}}
\addtocontents{toc}{\cftsetindents{subsubsection}{0cm}{1.5cm}}
\addtocontents{toc}{\protect\renewcommand{\protect\cftchapteraftersnum}{\enskip\textemdash\enskip}}


\printbibliography[title={REFERÊNCIAS}, heading=bibintoc] %

% ---
% Inicia os apêndices
% Para as seções nos apêndices usar o comando \section* que não adiciona a seção no sumário.
% ---
\appendix
\chapterstyle{apendices}

% ----------------------------------------------------------
\chapter{Processo de estabilização em cada passo}\label{passo}
% ----------------------------------------------------------
\lipsum[10]

% ----------------------------------------------------------
\chapter{Visualizações das pilhas de areia}\label{caleidoscopios}
% ----------------------------------------------------------
\lipsum[10]


\section*{Visualização da característica fractal da pilha de areia}
\lipsum[10]

% ---
% Inicia os anexos
% ---
\addtocontents{toc}{\protect\renewcommand\protect\cftappendixname{\nomeanexo~}}
\appendix
\chapterstyle{anexos}

% Imprime uma página indicando o início dos apêndices

% ----------------------------------------------------------
\chapter{Anexos}
% ----------------------------------------------------------
\lipsum[5]

% ----------------------------------------------------------
\chapter{outro anexos}

\lipsum[5]

\section*{seção no anexo}

\lipsum[5]

\section*{Outra seção no anexo}

\lipsum[5]

