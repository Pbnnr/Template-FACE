\chapter{Introdução}

\section{Figuras e Tabelas}
Nos elementos flutuantes, as legendas devem estar alinhadas à esquerda com o comando \textbf{minipage}.

\begin{verbatim}
\begin{table}[!ht]
    \centering
    \begin{minipage}{0.7\textwidth}     <<<<<<<
        \caption{\label{tabela:ComparativoFrameworks}
        Comparação dos Frameworks}
        \resizebox{\textwidth}{!}{
            [...]
        }
        \caption*{\footnotesize Fonte: Elaborado pelo autor, 2023.}
    \end{minipage}
\end{table}
\end{verbatim}

\begin{table}[!ht]
    \centering
    \begin{minipage}{0.7\textwidth}
        \caption{\label{tabela:ComparativoFrameworks}
            Comparação dos Frameworks}
        \resizebox{\textwidth}{!}{
            \begin{tabular}{|l|c|c|c|}
                \hline
                                       & \textbf{MapReduce} & \textbf{Spark} & \textbf{Flink} \\
                \hline
                \textbf{Armazenamento} & Disco              & RAM            & RAM            \\
                \hline
                \textbf{Granularidade} & Grossa             & Grossa         & Fina           \\
                \hline
                \textbf{Estado}        & Sem                & Sem            & Com            \\
                \hline
                \textbf{Processamento} & Lote               & Micro lotes    & Stream         \\
                \hline
                \textbf{Volume}        & Finito             & Finito         & Infinito       \\
                \hline
                \textbf{Linguagem.}    & Java               & Scala          & Java           \\
                \hline
            \end{tabular}
        }
        \caption*{\footnotesize Fonte: Elaborado pelo autor, 2023.}
    \end{minipage}
\end{table}

\section{Citação}

\subsection{Final do texto}
\begin{verbatim}
No atual cenário tecnológico, dados se tornaram um ativo de 
alto valor \cite{gunther2017debating}.
\end{verbatim}

No atual cenário tecnológico, dados se tornaram um ativo de alto valor \cite{gunther2017debating}.

\subsection{Início do texto}
\begin{verbatim}
Segundo \textcite{gunther2017debating} Dados se tornaram um 
ativo de alto valor no atual cenário tecnológico.
\end{verbatim}

Segundo \textcite{gunther2017debating} Dados se tornaram um ativo de alto valor no atual cenário tecnológico.

\section{Alíneas}
Para criar alíneas utilize o comando enumerate, nunca description ou itemize. As alínea devem encerrar com um ponto e vírgula e a última deve encerrar com um ponto final.

\begin{verbatim}
\begin{enumerate}
    \item Primeiro item da alínea;
    \item Segundo item da alínea;
    \item Terceiro item da alínea.
    \begin{enumerate}
        \item Primeiro item da subalínea;
        \item Segundo item da subalínea;
        \item Terceiro item da subalínea.
    \end{enumerate}
\end{enumerate}
\end{verbatim}

\begin{enumerate}
    \item Primeiro item da alínea;
    \item Segundo item da alínea;
    \item Terceiro item da alínea.
          \begin{enumerate}
              \item Primeiro item da subalínea;
              \item Segundo item da subalínea;
              \item Terceiro item da subalínea.
          \end{enumerate}
\end{enumerate}

\section{Customização}

Esse pacote pode ser customizado passando argumentos da seguinte forma:
\begin{verbatim}
\usepackage[acronym, glossaries, index, labelref, debug]{CEFET}
\end{verbatim}

\begin{enumerate}
    \item \textbf{acronym:} adiciona o suporte para lista de abreviaturas e siglas;
    \item \textbf{glossaries:} adiciona o suporte para glossário;
    \item \textbf{index:} adiciona o suporte para índice de assunto;
    \item \textbf{labelref:} \verb|\ref{fig:1}| retorna Figura 1 em vez de 1 para todas as referências;
    \item \textbf{debug:} Ativa as réguas e os quadros para melhorar a visualização das medidas.
\end{enumerate}

\section{Namedref}
\begin{verbatim}
Exemplo de utilização do \textbf{namedref} para a 
\ref{tabela:ComparativoFrameworks}.
\end{verbatim}
Exemplo de utilização do \textbf{namedref} para a \ref{tabela:ComparativoFrameworks}.