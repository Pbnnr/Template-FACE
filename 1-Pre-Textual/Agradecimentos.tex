\chapter*{Agradecimentos}
Qual a diferença entre dedicatória e agradecimento? 

A dedicatória na maioria das vezes é um texto curto, sucinto e bem objetivo que destaca a pessoa ou pessoas mais importantes na sua vida. Quando relacionada a vida pessoal, o autor deve agradecer a sua esposa, esposo, filhos, mãe, pai e avós.

No caso dos agradecimentos você não precisa se preocupar com o tamanho do texto. Você pode escrever um pouco mais sobre as pessoas que foram essenciais para seu sucesso. Nos agradecimentos, o autor pode falar da instituição, professores, coordenadores e amigos.

\section*{Exemplo: \href{https://www.abcm.org.br/downloads/TCC_-_Engenharia_Mecanica_2011-1_-_Adriano_Menezes_da_Silva.pdf}{\underline{link}}}

Agradeço primeiramente ao professor Msc. Dilson José Aguiar de Souza pela oportunidade de me orientar na conclusão deste trabalho e me ajudar na realização dos ensaios, além de me auxiliar com muita paciência.

Aos meus pais, Rubem Farias da Silva e Regina Cirinéia Menezes da Silva, por terem me dado força e sustentabilidade financeira no início do curso para chegar a esse momento. Aproveito também a oportunidade para agradecer todo o aporte que me deram em casa e o amor dedicado.

Aos meus irmãos Ana Paula Menezes da Silva e Alexandre Menezes da Silva pelas oportunidades de aprendizagem e troca de experiências.

À minha namorada Nicole Luise Fröehlich Kunsler pela dedicação oferecida, pelos momentos de companheirismo e pela compreensão aos momentos de ausência.

À empresa BLEISTAHL BRASIL METALURGIA S/A, em especial ao funcionário Manfred Kunrath, pela oportunidade de realizar o trabalho de conclusão com materiais fornecidos pela empresa, além de dar aporte financeiro para aquisição de materiais de apoio para a realização dos ensaios.

À empresa LESI Comércio e Representações LTDA, em especial a Fernando Mattes, representante na região da empresa SECO TOOLS que cedeu as ferramentas de corte para os ensaios.

Agradeço à UNISINOS pela cessão dos laboratórios da universidade e ao corpo de funcionários da casa, principalmente aos que me deram apoio e auxílio quando possível e sempre que necessário.